\documentclass[11pt]{article}
\usepackage[french]{babel}
\usepackage[utf8]{inputenc}
\usepackage[T1]{fontenc}


\newcommand{\TikZ}{Ti\textit{k}Z}

\begin{document}

\title{Schémas et graphiques en \LaTeX{} avec \TikZ{} - Notes pour la présentation}
\author{David Sandoz}
\date{\today}
\maketitle

\section{Introduction}
Ce cours a pour but de vous montrer de nombreuses possibilités de \TikZ{} et vous donnera une idée de comment s'y prendre. Toutefois il vous sera sûrement nécessaire de consulter les références pour pouvoir faire exactement ce que vous voudrez faire avec \TikZ.
\section{Figures simples}
\section{Graphiques}
\section{Diagrammes}
\section{Graphes}


\end{document}

\documentclass[12pt]{article}

\usepackage[french]{babel}
\usepackage[utf8]{inputenc}
\usepackage[T1]{fontenc}

\usepackage{fancyhdr}

\usepackage{multicol}
\setlength{\columnsep}{1cm}

%\usepackage[papersize={210mm,297mm}, left=2cm,right=2cm,top=2cm,bottom=2cm]{geometry}

\begin{document}

%header
%\pagestyle{fancy} %Ne pas oublier
%\lhead{Ce qui va \\ à gauche}
%\chead{Ce qui va \\ \textbf{au centre} du header}
%\rhead{Ce qui va à droite \\ $E = mc^{2}$}

%\lfoot{Ce qui va \\ à gauche}
%\cfoot{Ce qui va \\ \textbf{au centre} du footer}
%\rfoot{Ce qui va à droite \\ $E = mc^{42}$}

%\begin{multicols}{3}
%[
\section*{Le cormoran}
Les Phalacrocoracidae sont une famille d'oiseaux aquatiques constituée de 3 genres et de 36 espèces vivantes. Cette famille est celle des oiseaux de mer connus sous le nom de cormorans. Il s'agit d'un groupe très homogène, tant pour ce qui concerne la morphologie que le régime alimentaire piscivore, ou les mœurs générales. Pour cette raison, les 36 espèces de cormorans sont le plus souvent réunies dans un genre unique, le genre Phalacrocorax éponyme de la famille.
%]

\subsection*{Classification}

En dépit d'études récentes préconisant la distinction de trois genres (Microcarbo, Phalacrocorax et Leucocarbo), toutes les listes internationales actuelles — à l'exception de celle du Congrès ornithologique international qui valide cette recommandation — ne reconnaissent que le genre Phalacrocorax. Antérieurement, les classifications ont reconnu les genres Nannopterum (pour Nannopterum harrisi, maintenant le Cormoran aptère) et Leucocarbo (pour Leucocarbo atriceps, le Cormoran impérial).

\subsection*{Description}

Ce sont des oiseaux aquatiques, de taille moyenne à grande (de 45 à 100 cm), au corps allongé, au long cou et au bec puissant et crochu. Les cormorans arborent généralement un plumage noir et un long cou flexible. Ils pèsent de 1,5 à 3,5 kilogrammes.

\subsection*{Plumage et perméabilité}

Le plumage du cormoran est partiellement perméable, du fait que les trois quarts de la surface de sa plume (la partie la plus externe) ne comportent pas de crochets sur les barbules, rendant celles-ci libres et perméables. Cette propriété lui permet de dépenser moins d'énergie pour plonger, car moins d'air est emmagasiné dans son plumage que chez les autres oiseaux aquatiques, le rendant ainsi plus lourd.

La position du cormoran, ailes déployées à la sortie de l’eau lui permet entre autres de sécher ses plumes. Il s’avère que ce comportement permet également au cormoran une meilleure thermoregulation et facilite sa digestion.

\subsection*{Comportement}

En surface, il nage avec le corps très enfoncé, de sorte que, de loin, on ne voit dépasser que son cou. Très à l'aise sous l'eau, il peut nager en apnée jusqu'à une quarantaine de mètres de profondeur pendant plus de deux minutes, mais en général, il n'excède pas les dix mètres pour des plongeons d'une trentaine de secondes.

Il se déplace sous l'eau avec vélocité afin de capturer ses proies  : les poissons, aussi bien de mer que d'eau douce. La loi autorise en France l'abattage par des personnes accréditées de 32 000 Grands Cormorans. Mais cela n'influe en rien sur le nombre d'oiseaux, qui se stabilise naturellement.

\subsection*{Habitat}

Cosmopolites, avec la plus grande diversité en régions tropicales, tempérées, observables même en régions australes (Terre de Feu), en arctique et en antarctique, les cormorans fréquentent les étendues d'eau libre, à la fois sur les côtes et à l'intérieur des terres.

Les espèces les plus courantes en France sont le Grand Cormoran, une espèce marine que l'on retrouve également sur les fleuves, rivières et plans d'eau à l'intérieur des terres, et le cormoran huppé qui est une espèce exclusivement marine. À l'exception de quelques espèces pélagiques, le cormoran ne fréquente pas la haute mer et vit en général le long des côtes rocheuses et des falaises. On le retrouve sur presque tous les continents. Néanmoins, certaines espèces sont en voie de disparition ou protégées.

Sa grande diversité d'habitat l'oblige à s'adapter, mais c'est son comportement qui change et non sa physiologie (il ne baisse pas sa température corporelle). Ainsi, le Grand Cormoran vivant au Groenland mange deux fois plus en hiver que celui vivant en Normandie. Les cormorans réussissent également la prouesse de pêcher dans des eaux obscures, ce qui laisse planer des questions concernant d'éventuels moyens de détection autres que visuels.

%\end{multicols}



\end{document}
